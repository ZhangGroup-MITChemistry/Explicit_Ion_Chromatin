\documentclass[aip,jcp, preprint, amssymb, amsmath]{revtex4-1}
\usepackage{amssymb, amsmath,amsfonts}
\usepackage{graphicx}% Include figure files
\usepackage{dcolumn}% Align table columns on decimal point
\usepackage{bm}% bold math
\usepackage{multirow}
\usepackage{natbib}
\usepackage{hyperref}
\usepackage{url}
%\usepackage{float}
%\restylefloat{table}

\begin{document}

\newcommand{\rmsub}[2]{\ensuremath{#1_{\rm #2}}}
\newcommand{\rmsubsub}[3]{\ensuremath{{#1_{\rm #2}}_{#3}}}
\newcommand{\degree}{\ensuremath{^\circ}}

\title{Erratum: "An Experimentally-Informed Coarse-Grained 3-Site-Per-Nucleotide Model of DNA: Structure, Thermodynamics, and Dynamics of Hybridization." [J. Chem. Phys. 139, 144903 (2013)]}

\author{Daniel M. Hinckley}
\author{Gordon S. Freeman}
\affiliation{Department of Chemical and Biological Engineering\\ University of Wisconsin--Madison, Madison, WI, 53706}
\author{Jonathan K. Whitmer}
\affiliation{Institute for Molecular Engineering\\ Argonne National Laboratory, Chicago, IL, 60439}
\author{Juan J. de Pablo}
\email{depablo@uchicago.edu}
\affiliation{Institute for Molecular Engineering\\ University of Chicago, Chicago, IL, 60637}

\date{\today}

\begin{abstract}
\end{abstract}

\maketitle

Table III of our paper\cite{Hinckley2013} reported incorrect values for $\sigma$ of the S site and the $x$ coordinate of the C site.
The corrected table is shown below as Table \ref{topo-table} with the corrected value in bold.  
This change also requires slight changes to Tables IV, V, VI, and VII of the original paper.
The corrected versions of these Tables can also be found below.

This change to our paper does not invalidate the results found therein.
The 3SPN.2 DNA model penalizes deviations from a reference structure; 
as the present change involves changing an $x$ coordinate by $-0.546$\ \AA, the nature of the deviations will be essentially identical.
%As long as the topology and potential functions are self-consistent, similar behavior will result.

% New coordinate Table III
\begin{table}

\begin{ruledtabular}
\caption{Cartesian and cylindrical polar coordinates for the 3SPN.2 representation of DNA.
The masses and the excluded volume diameters of each site are also included.
The molecular topology of a single strand is built from the 3' end using a transformation directly related to the helical rise (3.38\AA\ and twist (36\degree) of B-DNA.
For example, if a sugar site is placed at ($r$, $\phi$, and $z$), the next sugar site moving in the 3' to 5' direction will be placed at ($r$, $\phi+ 36\degree$, and $z +3.38$\AA.
The sites of the complementary strands are related by a dyad along the x-axis; for a sugar site at ($x$,$y$,$z$), the sugar site of the complementary nucleotide will be located at $x$, $-y$, $-z$.
For base sites, the values of $r$, $\phi$, $x$, and $y$, and $z$ that are used should correspond to the identity of the site being placed.
For additional details, see Ref. \onlinecite{Arnott1976}.}
\label{topo-table}
\begin{tabular}{lccccccc}
Site Type & $x$ & $y$  & $z$ & $r$ & $\phi$ & $m$ & $\sigma$  \\
 & (\AA)& (\AA)& (\AA) & (\AA) & (\degree) & (amu) & (\AA) \\ \hline
Phosphate (P) & -0.628 & 8.896 & 2.186 & 8.918 & 94.035 & 94.97 & 4.5 \\
Sugar (S)  & 2.365 & 6.568 & 1.280 & 6.981 & 70.196 & 83.11 & \textbf{6.2} \\
Adenine (A) & 0.296 & 2.489 & 0.204 & 2.506 & 83.207 & 134.1 & 5.4 \\
Thymine (T) & -0.198 & 3.412 & 0.272 & 3.418 & 93.327 & 125.1 & 7.1 \\
Guanine (G) & 0.542 & 2.232 & 0.186 & 2.297 & 76.349 & 150.1 & 4.9 \\
Cytosine (C) & \textbf{0.137} & 3.265 & 0.264 & 3.336 & 78.192 & 110.1 & 6.4 \\ \hline\\
\end{tabular}
\end{ruledtabular}
\end{table}

% New Table IV
\begin{table}
\begin{ruledtabular}
 \caption{Table of 3SPN.2 force field parameters used in the bonded and non-bonded interactions.\label{Parameters}}

\begin{tabular}{|l|c|}
\textbf{Parameter} & \textbf{Value}  \\ \hline
$k_b$ & 0.6 kJ/mol/\AA$^2$ \\
$k_\theta$ & 200 kJ/mol/rad$^2$ \\
$k_\phi$ & 6.0 kJ/mol \\
$\epsilon_\text{r}$ & 1.0 kJ/mol \\
$K_{BS}$ & 6.0\\
$\alpha_{BS}$ & 3.0 \\
$K_{CS}$ & 8.0 \\
$\alpha_{CS}$ & 4.0 \\
$K_{BP}$ & 12.0 \\
$\alpha_{BP}$ & 2.0\\
$\sigma_{AT}$ & 5.941 \AA\\
$\sigma_{GC}$ & \textbf{5.530} \AA\\
$\epsilon_{AT}$ & 16.73 kJ/mol\\
$\epsilon_{GC}$ & 21.18 kJ/mol\\
\end{tabular}
\end{ruledtabular}
\end{table}

% New Table V
\begin{table}[H]
\begin{ruledtabular}
\caption{Equilibrium bond lengths $r_o$, bend angles $\theta_o$, and dihedral angles $\phi_o$.
The direction of the bonds is important.
P(5') or S(5') represents the phosphate or sugar, respectively, in the 5' direction of the adjacent site while P(3') or S(3') represents the phosphate or sugar in the 3' direction.}
\label{BondedParameters}
\begin{tabular}{lc}

% Bonds
\begin{tabular}{l}
\begin{tabular}{lc}
 Bond & $r_o$ (\AA) \\ \hline
P(5')--S & 3.899  \\
S--P(3') & 3.559\\
S--A & 4.670 \\
S--T & 4.189\\
S--G & 4.829\\
S--C & \textbf{4.112}\\
\end{tabular}
\\
\begin{tabular}{lcc}
 Dihedrals& $\phi_o$ ($^{\mbox{\tiny{o}}}$)& $\sigma_{\phi}$ \\ \hline
(5')P--S--P--S(3') & -154.79 & 0.30 \\
(5')S--P--S--P(3') & -179.17 & 0.30 \\
\end{tabular}
\end{tabular}

% Angles
&
\begin{tabular}{lc}
Bend & $\theta_o$ ($^{\mbox{\tiny{o}}}$) \\ \hline
S--P--S & 94.49 \\
P--S--P  & 120.15\\
P--S--A & 103.53\\
P--S--T & 92.06\\
P--S--G & 107.40\\
P--S--C & \textbf{96.96}\\
A--S--P & 112.07\\
T--S--P & 116.68\\
G--S--P & 110.12\\
C--S--P & \textbf{114.34}\\
\end{tabular}


\end{tabular}
\end{ruledtabular}
\end{table}

% Table VI
\begin{table*}
\begin{ruledtabular}
\caption{Reference angles used to modulate \rmsub{U}{bp} and \rmsub{U}{cstk}.
The indices $i$ and $j$ correspond to the identity of the base sites being used to define the vector $r_{ij}$.
All angles are expressed in degrees.
}\label{base-base-angles}
\begin{tabular}{ccc}
& \multicolumn{2}{c}{Base $j$} \\
& \rmsubsub{\phi}{1}{o} & \rmsubsub{\theta}{1}{o}  \\
\multirow{3}{*}{Base $i$} &
\begin{tabular}{c|cccc}
  & A & T & G & C \\ \hline
A & -- & -38.35 & -- & -- \\
T & -38.35 & -- & -- & -- \\
G & -- & -- & -- & \textbf{-42.98} \\
C & -- & -- & \textbf{-42.98} & -- \\
\end{tabular} &
\begin{tabular}{c|cccc}
  & A & T & G & C \\ \hline
A & -- & 156.54 & -- & -- \\
T & 135.78 & -- & -- & -- \\
G & -- & -- & -- & \textbf{159.81} \\
C & -- & -- & \textbf{141.16} & -- \\
\end{tabular} \\ \\

& \rmsubsub{\theta}{2}{o} & \rmsubsub{\theta}{3}{o} \\
 &
\begin{tabular}{c|cccc}
  & A & T & G & C \\ \hline
A & -- & 135.78 & -- & -- \\
T & 156.54 & -- & -- & -- \\
G & -- & -- & -- & \textbf{141.16} \\
C & -- & -- & \textbf{159.81} & -- \\
\end{tabular} &
\begin{tabular}{c|cccc}
  & A & T & G & C \\ \hline
A & -- & 116.09 & -- & -- \\
T & 116.09 & -- & -- & -- \\
G & -- & -- & -- & \textbf{124.94} \\
C & -- & -- & \textbf{124.94} & -- \\
\end{tabular}
\end{tabular}

\end{ruledtabular}
\end{table*}


% Table VII
\begin{table*}
\begin{ruledtabular}
\caption{Values of the strengths $\epsilon_{ij}$, equilibrium distances $\sigma_{ij}$, and equilibrium angles \rmsubsub{\theta}{XX}{o} for the base stacking (a) and cross-stacking interactions (b-c).
The arrows $\uparrow$ and $\downarrow$ represent the sense and anti-sense strands, respectively with the bases participating in the base pair indicated by $^{5'}\uparrow$ and $\downarrow^{3'}$.
$_{3'}\uparrow$ and $\downarrow_{5'}$ indicate adjacent bases in the 3' and 5' directions, respectively, that participate in cross-stacking interactions.
}\label{base-parameters}
\begin{tabular}{rc}
(a) &
\begin{tabular}{cccc}
& \multicolumn{3}{c}{Base $_{3'}\uparrow$} \\
& $\epsilon$ & $\sigma$ & \rmsubsub{\theta}{BS}{o} \\
& (kJ/mol) & (\AA) & ($^{\mbox{\tiny{o}}}$) \\
\text{Base} $^{5'}\uparrow$ &
\begin{tabular}{c|cccc}
  & A & T & G & C \\ \hline
A & 14.39 &  14.34 &  13.25 &  15.51 \\
T & 10.37 &  13.36 &  10.34 &  12.89 \\
G & 14.81 &  15.57 &  14.93 &  15.39 \\
C & 11.42 &  12.79 &  10.52 &  13.24 \\
\end{tabular} &
\begin{tabular}{c|cccc}
  & A & T & G & C \\ \hline
A & 3.716 & 3.675 & 3.827 & \textbf{3.744} \\
T & 4.238 & 3.984 & 4.416 & \textbf{4.141} \\
G & 3.576 & 3.598 & 3.664 & \textbf{3.635} \\
C & \textbf{4.038} & \textbf{3.798} & \textbf{4.208} & \textbf{3.935} \\
\end{tabular} &
\begin{tabular}{c|cccc}
  & A & T & G & C \\ \hline
A & 101.15 & 85.94 & 105.26 & \textbf{89.00} \\
T & 101.59 & 89.50 & 104.31 & \textbf{91.28} \\
G & 100.89 & 84.83 & 105.48 & \textbf{88.28} \\
C & \textbf{106.49} & \textbf{93.31} & \textbf{109.54} & \textbf{95.46} \\
\end{tabular}
\end{tabular}\\ \\ \hline \\
(b) &
\begin{tabular}{cccc}
& \multicolumn{3}{c}{Base $\downarrow_{5'}$} \\
& $\epsilon$ & $\sigma$ & \rmsubsub{\theta}{CS}{o} \\
& (kJ/mol) & (\AA) & ($^{\mbox{\tiny{o}}}$) \\
\text{Base} $^{5'}\uparrow$ &
\begin{tabular}{c|cccc}
  & A & T & G & C \\ \hline
A & 2.186 &  2.774 &  2.833 &  1.951 \\
T & 2.774 &  2.186 &  2.539 &  2.980 \\
G & 2.833 &  2.539 &  3.774 &  1.129 \\
C & 1.951 &  2.980 &  1.129 &  4.802 \\
\end{tabular} &
\begin{tabular}{c|cccc}
  & A & T & G & C \\ \hline
A & 6.208 & 6.876 & 6.072 & \textbf{6.811} \\
T & 6.876 & 7.480 & 6.771 & \textbf{7.453} \\
G & 6.072 & 6.771 & 5.921 & \textbf{6.688} \\
C & \textbf{6.811} & \textbf{7.453} & \textbf{6.688} & \textbf{7.409} \\
\end{tabular} &
\begin{tabular}{c|cccc}
  & A & T & G & C \\ \hline
A & 154.38 & 159.10 & 152.46 & \textbf{158.38} \\
T & 147.10 & 153.79 & 144.44 & \textbf{151.48} \\
G & 154.69 & 157.83 & 153.43 & \textbf{158.04} \\
C & \textbf{152.99} & \textbf{159.08} & \textbf{150.53} & \textbf{157.17} \\
\end{tabular}
\end{tabular} \\ \\ \hline \\

(c) &
\begin{tabular}{cccc}
& \multicolumn{3}{c}{\text{Base} $_{3'}\uparrow$} \\
& $\epsilon$ & $\sigma$ & \rmsubsub{\theta}{CS}{o} \\
& (kJ/mol) & (\AA) & ($^{\mbox{\tiny{o}}}$) \\
Base $\downarrow^{3'}$ &
\begin{tabular}{c|cccc}
  & A & T & G & C \\ \hline
A & 2.186 &  2.774 &  2.980 &  2.539 \\
T & 2.774 &  2.186 &  1.951 &  2.833 \\
G & 2.980 &  1.951 &  4.802 &  1.129 \\
C & 2.539 &  2.833 &  1.129 &  3.774 \\
\end{tabular} &
\begin{tabular}{c|cccc}
  & A & T & G & C \\ \hline
A & 5.435 & 6.295 & 5.183 & \textbf{6.082} \\
T & 6.295 & 7.195 & 6.028 & \textbf{6.981} \\
G & 5.183 & 6.028 & 4.934 & \textbf{5.811} \\
C & \textbf{6.082} & \textbf{6.981} & \textbf{5.811} & \textbf{6.757} \\
\end{tabular} &
\begin{tabular}{c|cccc}
  & A & T & G & C \\ \hline
A & 116.88 & 121.74 & 114.23 & \textbf{119.06} \\
T & 109.42 & 112.95 & 107.32 & \textbf{110.56} \\
G & 119.34 & 124.72 & 116.51 & \textbf{121.98} \\
C & \textbf{114.60} & \textbf{118.26} & \textbf{112.45} & \textbf{115.88} \\
\end{tabular}
\end{tabular} \\ \\

\end{tabular}
\end{ruledtabular}
\end{table*}

% Table with relevant angles

% phi theta1 theta2 theta3

\bibliography{erratum}
\end{document}
